\documentclass{article}

% Language setting
% Replace `english' with e.g. `spanish' to change the document language
\usepackage[english]{babel}

% Set page size and margins
% Replace `letterpaper' with `a4paper' for UK/EU standard size
\usepackage[a4paper,top=2cm,bottom=2cm,left=3cm,right=3cm,marginparwidth=1.75cm]{geometry}

% Useful packages
\usepackage{amsmath}
\usepackage{graphicx}
\usepackage[colorlinks=true, allcolors=blue]{hyperref}

\usepackage[T1]{fontenc}
\usepackage{tgbonum}

\usepackage{enumitem}

\renewcommand{\thesection}{\Roman{section}}
\renewcommand{\thesubsection}{\thesection.\Roman{subsection}}
\renewcommand{\thesubsubsection}{\thesection.\alph{subsubsection}}

\newenvironment{subs}
  {\adjustwidth{3em}{0pt}}
  {\endadjustwidth}

\title{\textbf{Evaluation Protocol}}
\author{Abhiraman Kuntimaddi}
\date{February 25, 2022}

\begin{document}
\maketitle

\section{General Outline}

\subsection{Evaluation Goals}

Analyze an XR(Extended Reality) application in Cross Platform Framework Development (CPFD) 
like OpenXR using Mixed Reality Feature Tool and Native implementation like Mixed Reality ToolKit(MRTK)
on Unity Game Engine for the purpose of comparision with respect to Effectiveness, Effiency and Satisfaction.

\subsection{Tools to Compare}

\begin{enumerate}[label=\textbf{\alph*}]
	\item MRTK(Mixed Reality ToolKit) Native implementation in Unity
	\item MRTK OpenXR(Cross Platform Framework Development) implementation in Unity
\end{enumerate}

\subsection{Scenarios}

Described in further details in the Evaluation Task sheet.

\subsection{Variables}

\begin{enumerate}[label=\textbf{\alph*}]
	\item Dependent Variables : Effictiveness and Accuray
	\item Effectiveness is to be measured in terms of the accuracy (closeness of measurements of a quantity to that quantity’s actual/true value) with which certain tasks are performed.
	\item Efficiency is to be measured in terms of required time to accomplish the tasks chosen from the task sheet.
	\item Performance Expectancy and Effort Expectancy are to be used for rating Acceptability.
	\item Utility, Intuitiveness, Learnability, and Personal Effect are to be used for rating Usability.
	\item Usability (Questionnaire closed-ended)
	\item Acceptance to be measured using TAM (Technology Acceptance Model; validated and standardized test instrument).
	\item Independent variable: XR Devices and Tools
\end{enumerate}

\subsection{Target sample}
\begin{enumerate}[label=\textbf{\alph*}]
	\item Knowledgeable with respect to Virtual Enviroments (XR/MR/AR/VR) and Unity Game Engine.
	\item Convenient Sample : Researchers and Students of Apl. Prof. Dr. Achim Ebert and Dr.-ing. Taimur Kausar Khan.
\end{enumerate}

\subsection{Experimental Setup}
\begin{enumerate}[label=\textbf{\alph*}]
	\item General Design : Quasi-experiment.
	\item Assignment of participants to group : Both Native and OpenXR(CPFD) implementations are very similar, Both groups will solve similar but on different days for avoiding bias.
\end{enumerate}

\section{Hypothesis}
Careful consideration of the visualization and appropriate interactions in the virtaul world with XR Devices and OpenXR seem to be important for understanding the difference between Native and CPFD
The comparisions are captured in the following Hypothesis :

\begin{itemize}
	\item \textbf{Research Hypothesis H1} \hfill \break
	      \textit{The use of Cross Platform Framework Development APIs such as OpenXR in game engines like Unity has an impact on Effectiveness, Efficiency and Satisfaction when compared to that of Native API implementations like MRTK.}
  \item \textbf{Research Hypothesis H2} \hfill \break
        \textit{Deciding over which Cross Platform Framework Development ToolKit turns out to be the best choice in the concrete application.}
\end{itemize}

\section{Operationalization}


% \bibliographystyle{alpha}
% \bibliography{sample}

\end{document}